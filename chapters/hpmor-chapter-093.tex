\partchapter{Roles}{IV}

\lettrine{H}{arry} had walked into the Great Hall, looked around only once, grabbed enough calories to sustain himself, walked out, put on his Cloak again and found a small random corner in which to eat. Seeing the students at their tables—

\emph{Feeling revulsion when you look at other humans is not a good sign,} Hufflepuff said. \emph{It’s not reasonable to blame them for having not had your opportunities to learn what you’ve learned. Inaction in emergencies has nothing to do with people being selfish. Normalcy bias, like that plane crash in Tener-something where a few people ran out and escaped but most people just sat in their seats not moving while their plane was literally on fire. Look at how long} you \emph{took to really start moving.}

\emph{It serves no useful purpose to hate}, said Gryffindor. \emph{It’s just going to damage your altruism.}

\emph{Try to figure out a training method you could use to prevent this from happening next time,} said Ravenclaw.

\emph{I’ll go ahead and register the experimental prediction,} said Slytherin, \emph{that we’ll always observe exactly what would be predicted on the hypothesis that people cannot be saved, cannot be taught, and will never help us with anything important. Also, we need some way of keeping track of all the times I’m right.}

Harry ignored the voices in his head and just ate slices of toast as fast as he could. It wasn’t proper nutrition as a general policy, but one-time exceptions wouldn’t hurt so long as he made them up the next day.

In mid-bite, the blazing silver silhouette of a phoenix flew in from nowhere and said, in the voice of a tired old man, “Please remove your Cloak, Harry, I have a letter to deliver to you.”

Harry coughed for a bit, swallowed some toast which had gone down the wrong way, stood up, took off the Cloak of Invisibility, said aloud “Tell Dumbledore I said fine,” and then sat down and continued to eat his toast.

The toast had all gone by the time Albus Dumbledore walked up to Harry’s nook, carrying folded sheets of paper in his hand; real paper, with lines, not wizard’s parchment.

“Is that—” Harry said.

“From your father, and from your mother,” said the old wizard. Wordlessly, Dumbledore handed over the folded sheets, and wordlessly Harry accepted them. The old wizard hesitated, then said quietly, “The Defense Professor has told me to restrain my counsel, and I thought the same thing myself when given time to think. I have always taken too long to learn the virtues of silence. But if I am mistaken, you need only say the word—”

“You’re not mistaken,” Harry said. He looked down at the folded, lined papers, feeling the sickness in his gut that was how his body indicated a strong pessimistic prediction. His parents wouldn’t actually disown him, and there wasn’t much they could \emph{do} to him (some part of himself was still afraid in a very visceral way of television privileges being taken away, no matter how little sense that made now). But he had stepped outside the role that parents would expect of children who, in their internal beliefs, were lower on the pecking order. It would be stupid to expect anything except complete indignant fury, all-out righteous rage, when you acted like that to someone who thought they were dominant over you.

“After you read it,” the Headmaster said, “I believe that you should come to the Great Hall at once, Harry. There is an announcement which you will wish to hear.”

“I’m not interested in funerals—”

“No. Not that. Please, Harry, come as soon as you are done reading, and do so without your Cloak. Will you?”

“Yes.”

The old wizard left.

Harry had to force himself to open up the letter. The important thing was keeping your vulnerable friends and relations out of harm’s way, it might be a cliche but so far as Harry could tell the logic was valid. Damaged relationships could be repaired later.

The first letter said, in script handwriting that required a careful focus for Harry to read,

\emph{Son,}

\emph{No matter what you’ve read in books, keeping us out of harm’s way is} \emph{not} \emph{as important as having adults who can help when you’re in trouble. You decided without giving us a word in edgewise that we’d abandon you because of your ‘dark side’. The ghost of Shakespeare knows that I’ve seen things in this last year that were not dreamt of in my philosophy—sometimes I wonder if your Mum isn’t just humoring me and the authorities took you away when I started thinking you were a magic-user—so I can’t deny that it’s} \emph{possible} \emph{you’ve managed to develop some... I’m not quite sure what to call it, but ‘dark side’ seems premature if we don’t know what’s happening. Are you sure it’s not a burgeoning telepathic talent and you’re just picking up on the minds of other wizards around you? Their thoughts might seem evil to a child who grew up in a saner civilization. These are ungrounded speculations, I admit, but you shouldn’t jump to conclusions either.}

\emph{The two most important things I have to tell you are this. First, son, I have} \emph{every} \emph{confidence in your ability to stay on the Light Side of the Force so long as you choose to, and I have every confidence that you will choose to. If there’s some evil spirit whispering horrible suggestions in your ears, just ignore the suggestions. I} \emph{do} \emph{feel the need to emphasize that you should exercise special caution to ignore this evil spirit even if it is suggesting what seem like wonderful creative ideas and I hope I do not need to remind you about the Incident with the Science Project which would, I admit, make a deal more sense if you were struggling with demonic possession.}

\emph{The second thing I have to say is that you do not need to fear that Mum or I are going to abandon you because of your ‘dark side’. We may not have expected you to gain magical powers or develop an affinity for black magic, but we did expect you to become a teenager. Which, if you think about it from your poor father’s perspective, is already a sufficiently worrying prospect regarding a child who, by the age of nine, had been party to the summoning of a total of five fire engines. Children grow up. I won’t lie to you and say that you will feel as close to us at 20 as you do now. But your Mum and I will feel just as close to you when we are old and grey and bothering the nursing-home robots. Children always grow up and away from their parents, and the parents always follow them from behind, offering helpful advice. Children grow up, and their personalities change, and they do things that their parents wish they would not do, and they act disrespectfully toward their parents and have them hauled out of their magical schools, and the parents go on loving them anyway. It is Nature’s way. Though in the event that you have not yet hit puberty and your teenage years are proportionately worse than this, we reserve the right to reconsider this sentiment.}

\emph{No matter what is happening, remember that we love you and will always love you no matter what. I don’t know if our love has any magical power under your rules, but if it does, don’t hesitate to call on it.}

\emph{With all of this said... Harry, what you did there is not acceptable. I think you know that. And I also know that it is not the time to lecture you on it. But you must write and tell us what is happening. I can understand very well why you’d want us taken out of your school at once, and I know we can’t force you to do anything, but please, Harry, be reasonable and realize how terrified we must be.}

\emph{I would like to tell you that you are absolutely forbidden to mess around with any magic that the adults around you consider the least bit unsafe, but for all I know, the teachers at your school are giving everyone lessons in advanced necromancy every Monday. Please, please exercise as much caution as your situation permits, whatever your situation may be. Despite your very hurried summary we don’t have the slightest idea what is happening and I hope that you will write us as much as you can. It is clear that you are, at least in some ways, growing up, and I will} \emph{try} \emph{not to act like the children’s-book parent who only makes things worse—though I hope you appreciate how hard this is—and your Mum has said a number of frightening things to me about how wizardry stays secret and how I might get} \emph{you} \emph{into trouble by making waves. I cannot tell you to avoid anything unsafe, because your school is unsafe and your Headmaster will not let you leave. I can’t tell you that you shouldn’t take responsibility for anything happening around you, because for all I know there are other children in trouble. But remember that it is} \emph{not} \emph{your moral responsibility to protect any adults, their place is to protect you, and every good adult would agree with that. Please write and tell us more as soon as you can.}

\emph{Both of us are desperate to help. If there is anything at all that we can do, please let us know at once. There is nothing which can happen to us which would be worse than learning that something had happened to you.}

\emph{Love,<br>}\emph{Dad.}\emph{<br>}

The last page said only,

\emph{You promised me that you wouldn’t let magic take you away from me. I didn’t raise you to be a boy who would break a promise to his Mum. You must come back safely, because you promised.}

\emph{Love,<br>}\emph{Mum.}\emph{<br>}

Slowly, Harry lowered the letters and began to walk towards the Great Hall. His hands were shaking, his whole body was shaking, and it seemed to be taking a very great deal of effort not to cry; which he knew wordlessly that he must not do. He hadn’t cried through all of the day. And he wouldn’t cry. Crying was the same as admitting defeat. And this wasn’t over. So he wouldn’t cry.
\sbreak

The food served in the Great Hall that evening was plain that night, toast and butter and jam, water and orange juice, oatmeal and other simple fare, without dessert. Some students had worn simple black robes without their House colors. Others had still worn theirs. It should have been cause for argument, but there was instead a quietness, the sound of people eating without talking. It took two sides to make a debate, and one of the sides, this night, was not much interested in debating.

Deputy Headmistress Minerva McGonagall sat at the Head Table and did not eat. She should have. Perhaps she would in a short while. But she could not force herself to do it now.

For a Gryffindor there was only one path. It had taken Minerva only a short time to remember that, when after the Defense Professor’s urgings her mind had stayed empty of clever plots to try. That was not a Gryffindor’s way; or perhaps she ought to say only that it was not \emph{her} way, Albus did seem to try his hand at plotting... and yet when she thought back on their history, there were no plots at the moment of crisis, no cleverness and games in the last resort. For Albus Dumbledore, as for her, the rule \emph{in extremis} was to decide what was the right thing to do, and do it no matter the cost to yourself. Even if it meant breaking your bounds, or changing your role, or letting go of your picture of yourself. That was the last resort of Gryffindor.

Through a side entrance of the Great Hall she saw Harry Potter quietly slip in.

It was time.

Professor Minerva McGonagall rose from her chair, straightened the worn point on her hat, walked slowly to the lectern before the Head Table.

The sounds in the Great Hall, already muted, fell away entirely as all students turned to look at her.

“By now you have all heard,” she said, her voice not quite steady. \emph{That Hermione Granger is dead.} She didn’t say those words aloud, since they had all heard. “Somehow, a troll was infiltrated into the castle Hogwarts without alarm from our ancient wards. Somehow this troll succeeded in injuring a student, without alarm from the wards until the point of her death. Investigations are underway to determine how this has occurred. The Board of Governors is meeting to determine how Hogwarts will respond. In due time justice shall be served. Meanwhile there is another matter of justice, which must be handled at once. George Weasley, Fred Weasley, please come forward to stand before us all.”

The Weasley twins exchanged glances where they sat at the Gryffindor table, and then stood up and walked toward her, slowly, reluctantly; and Minerva realized then that the Weasley twins thought that they were to be expelled.

They honestly thought that she would expel them.

That was what the picture of Professor McGonagall who lived in her head had wrought.

The Weasley twins walked over to the lectern, looking up at her with faces that were frightened, but resolute; and she felt something in her heart break a little further.

“I am not going to expel you,” she said, and was saddened further by the surprised look on their faces. “Fred Weasley, George Weasley, turn and face your classmates, let them see you.”

Still looking surprised, the Weasley twins did so.

She drew up all the steel in her heart, and said what was right.

“I am ashamed,” said Minerva McGonagall, “of the events of this day. I am ashamed that there were only two of you. Ashamed of what I have done to Gryffindor. Of all the Houses, it should have been Gryffindor to help when Hermione Granger was in need, when Harry Potter called for the brave to aid him. It was true, a seventh-year could have held back a mountain troll while searching for Miss Granger. And you should have believed that the Head of House Gryffindor,” her voice broke, “would have believed in you. If you disobeyed her to do what was right, in events she had not foreseen. And the reason you did not believe this, is that I have never shown it to you. I did not believe in you. I did not believe in the virtues of Gryffindor itself. I tried to stamp out your defiance, instead of training your courage to wisdom. Whatever the Sorting Hat saw in me that led it to place me in Gryffindor, I have betrayed it. I have offered my resignation to the Headmaster as Deputy Headmistress and as the Head of House Gryffindor.”
\sbreak

There were cries of shock and dismay, and not only from the Gryffindor Table, as Harry’s heart froze within his chest. Harry needed to run forward, say something, he hadn’t meant for \emph{this} to—
\sbreak

Minerva took another breath, and continued. “However, the Headmaster has declined to accept my resignation,” she said. “So I will continue to serve, and try to undo what I have wrought. Somehow I must find a way to teach my students how to do what is right. Not what is safe, not what is easy, not what we are told to do. If all I can teach you is to turn in your essays on time, there might as well not be a House Gryffindor. This road will be more difficult for me, and maybe for all of us. But I know now that before I was only taking the easy path.”

She stepped down from the lectern, moved down to where the Weasley twins stood.

“Fred Weasley, George Weasley,” she said. “The two of you have not always done what is right. The path of wisdom does not lie in flagrant and needless defiance of authority. And yet today you proved to be the last of our House to survive my mistakes. Because it was the right thing to do, you defied a threat of expulsion and risked your lives to face a mountain troll. For your astounding courage that honors your House to have you, I award each of you two hundred points for Gryffindor.”

Again the look of shock on their faces, again the pain like a knife through her heart.

She turned to face the other students.

“I will not award any points to Ravenclaw,” she said. “I suspect that Mr.~Potter would not want them. If I am wrong, he may correct me and take as many House points as he pleases. But for whatever it is worth, Mr.~Potter, I am,” her voice faltered, “I am sorry—”
\sbreak

\emph{“Stop!”} Harry screamed, and then, again, “Stop.” The word sticking in his throat. “You don’t have to, Professor.” Something inside him was twisting, threatening to split him open, like a giant’s hands wrenching at him to tear him in half. “And, and you shouldn’t forget Susan Bones, and Ron Weasley—they also helped, they should get House points too—”

“Miss Bones and the young Weasley?” said Professor McGonagall. “Rubeus said nothing of that—what did they do?”

“Miss Bones tried to stun Mr.~Hagrid when he tried to stop me, and Mr.~Weasley shot Neville when Neville tried to stop me. They should both get points, and, and so should Neville,” Harry hadn’t thought to imagine it before, the way Neville must be feeling now, but the instant he’d thought, he knew, “because Neville tried to do something, even if it wasn’t the right thing, doing what’s right is the \emph{second} lesson, you can start practicing that after you learn to do anything at all—”

“Ten points to Hufflepuff, Miss Bones,” Professor McGonagall said, her voice breaking in the middle. “Ten points to Gryffindor, Ron Weasley, your family has done itself exceeding proud, this day. And ten points to Hufflepuff for Neville Longbottom, for standing up to Mr.~Potter and doing what he thought was right—”

\emph{“You shouldn’t!”} screamed a young voice from the Hufflepuff table, followed by a single choking sound.

Harry looked there, and then quickly looked back at Professor McGonagall and said, as steadily as he could, “Neville’s right, actually, you can’t award literally zero points for the part where you get the action correct, that sends the wrong message too, but he was halfway there so it could be five points instead.”

Professor McGonagall looked, for a moment, like she couldn’t think of what to say; but then her eyes went to Neville’s place at the table, and she said, “As you wish, Mr.~Potter. What is it, Miss Bones?”

Harry looked and saw that Susan Bones had stepped forward, wiping at her own eyes, and the Hufflepuff girl said, “Actually—Professor McGonagall—General Potter didn’t see it—but Captain Weasley and I weren’t the only ones who tried to get in Mr.~Hagrid’s way, after he ran out. Before some of the older students stopped us. But we managed to slow Mr.~Hagrid down a minute, so General Potter could get away.”

“You’ve got to give them points too,” said Ron Weasley from the Gryffindor table. “Or I won’t take any.”

“Who else?” said Professor McGonagall, her voice a bit unsteady.

Seven other children stood up.

\emph{What was that our Slytherin side was saying about predicting nothing would ever work?} said Hufflepuff.

Something in Harry cracked, so that he had to exert all his force to hold himself together.
\sbreak

When all had been said, and all had been done, Minerva went to where Harry Potter stood. Though it was not her greatest skill she cast a ward about them to blur vision, and muffled sounds with another thought.

“You, you didn’t have to—” said Harry Potter. “You shouldn’t have said—” He sounded like he was choking. “P-Professor, everything I said to you was hurtful, and hateful, and wrong—”

“I already knew that, Harry,” she said. “Even so, I wished to do better.” There was a feeling of lightness in her chest, much as one might experience after stepping off a cliff, when your legs no longer had to hold your body upright. She wasn’t sure she could do this, she did not know the way; and yet for the first time it seemed possible that Hogwarts wouldn’t become a sad ghost of its former self, when she became its Headmistress.

Harry stared at her, then made a odd noise that sounded like it had been forced from his throat, and covered his face in his hands.

So she knelt down, and hugged him. It might go wrong, but it might also go right, and she would not let that uncertainty stop her; it was time she began to learn a Gryffindor’s courage, so that she could teach it in turn.

“I had a sister once,” she whispered. Just that, and nothing more.
\sbreak

\emph{Just to make sure,} said some part of Harry, while the rest of him sobbed into Professor McGonagall’s arms, \emph{this doesn’t mean we’ve accepted Hermione’s death, right?}

\emph{NO} said all the rest of him, every part of his mind in unanimous agreement, warmth and cold and a hidden place of steel. \emph{Never, ever, forever.}
\sbreak

And an ancient wizard to whom that ward meant nothing gazed upon them both, the witch and the weeping young wizard. Albus Dumbledore was smiling with a strange sad look in his eyes, like someone who has taken one more step toward a foreseen destination.
\sbreak

The Defense Professor watched them both, the woman and the crying boy. His eyes were very cold, and very calculating.

He did not think that this would be enough.
\sbreak

It wasn’t until the next morning that it was discovered that Hermione Granger’s body was missing.
